%\documentclass{beamer}
\documentclass[hyperref={pagelabel=false,colorlinks=true,linkcolor=blue,citecolor=blue},12pt]{beamer}
\usepackage[T1]{fontenc}
\usepackage[latin1]{inputenc}
\usepackage{amsmath}
\usepackage{algorithm}
\usepackage{algorithmic}

\usetheme{Warsaw}

\title[Learning Music Representation using Recurrent Neural Networks]{Learning Music Representation using Recurrent Neural Networks}
\author{Cl�ment Jambou}
\institute{Imperial College of London, Departement of Computing}
\date{8 September 2014}


\begin{document}


\AtBeginSection[]
{
  \begin{frame}
  \tableofcontents[currentsection]
  \end{frame} 
}


\begin{frame}
\titlepage
\end{frame}


\begin{frame}
\tableofcontents

\end{frame}


\section{State of the art and motivation}

    \subsection{ State of the Art}

\begin{frame}
    \frametitle{Neural networks}

    \begin{itemize} 
        \item Feed Forward (FFN)
        \item Recurrent (RNN)
    \end{itemize}
    \begin{center}
        \begin{figure}
            \includegraphics[width=0.3\textwidth]{Figures/network.jpeg} 
            \hspace{1cm} \includegraphics[width=0.3\textwidth]{Figures/rnn.png} 
        \end{figure}
    \end{center}

\end{frame}

\begin{frame}
    \frametitle{model of RNN}
    Given a sequence of inputs: $x_1, x_2, ... x_t$ and the hidden states $h_1, h_2, ..., h_t$.
    $$
    \begin{array}{rcr} 
        h_0 & = & a(W_{hx}  * x_0 + b_{init} + b_h)  \\ 
        h_i & = & a(W_{hx}  * x_i + W _{hh} * h_{i-1} + b_h)  \\ 
        \hat{y}_i & = & a'(W_{yh} + b_y)

    \end{array}
    $$

\end{frame}


\begin{frame}
    \frametitle{Training : Backpropagation}
    Backpropagation = Gradient descent on loss. 

    \begin{itemize} 
        \item use of the derivative of composition product to go back layer after layer...
        \item backpropagation through time for RNNs
    \end{itemize}
    \begin{center}
        \begin{figure}
            \includegraphics[width=0.4\textwidth]{Figures/network.jpeg} 
        \end{figure}
    \end{center}

\end{frame}


\begin{frame}
    \frametitle{Vanishing gradient and Hessian Free Optimization}
    Backpropagation = Gradient descent on loss. 

    \begin{itemize} 
        \item Deep $\Rightarrow$ vanishing gradient (one more layer = one more Matrix multiplication)
        \item RNN : as deep as the number of timesteps. 
    \end{itemize}
    \begin{center}
        \begin{figure}
            \includegraphics[width=0.3\textwidth]{Figures/gradient_ellipse.jpg} 
        \end{figure}
    \end{center}

    Solution : Use second order information.
    Hessian Free Optimization use a quadratic approximation of the fitness function and performs the first steps of conjuguate gradient descent.

\end{frame}

\subsection{Motivation}

\begin{frame}
	\frametitle{Motivation:}
    \begin{itemize}
        \item Growth of computational power led to think : "The bigger network, the better."
        \item Current RNN structure uses dense Matrix where the brain connections map is sparse. 
    \end{itemize}
    
    \begin{itemize}

     \item Number of Neurons in human brain : $\sim 10^{11}$ 
     \item Number of Synapse in human brain : $\sim 10^{14}, 10 ^{15}$ 
    \end{itemize}
\end{frame}

\section{Study of RNNs}

 
\begin{frame}
    \frametitle{Experiment:}
    Pretrained RNN on a memory problem: 100 timesteps long sequences of bits. 
    Loss: Cross correlation of the output of the last timestep and the first input (Remember the first input)
    \begin{center}
         \begin{figure}
             \includegraphics[width=0.5\textwidth]{Figures/RNN.png}
        \end{figure}
    \end{center}
    Training : 8 hours, with Hessian Free Optimization

\end{frame}

  
\begin{frame}
    \frametitle{Plotting}
   Goal : Find a way of plotting the RNN.
    \begin{itemize}
        \item use weights as spring.
        \item spectral analyses of Laplacian matrix.
    \end{itemize}
    \begin{center}
         \begin{figure}
            \includegraphics[width=0.3\textwidth]{Figures/weighted_graph_spring.png} 
            \hspace{1cm} \includegraphics[width=0.3\textwidth]{Figures/weighted_graph_spectral.png} 
        \end{figure}
    \end{center}

\end{frame}

\begin{frame}
    \frametitle{Question : Are all the neurons useful?}
    We remove one neuron at a time (by replacing its input/output weights in Whh with 0) and test the network on dataset. 
    \begin{center}
         \begin{figure}
            \includegraphics[width=0.5\textwidth]{Figures/error_test_set_remove_neuron.png} 
        \end{figure}
    \end{center}
    Results : Neurons importance is very unequal (though the sum of absolute value of the weights is very constrained) 
    We order the neurons using this method to obtain the "worth" of a neuron.
\end{frame}


\begin{frame}
    \frametitle{Comparaison with classical metrics}
    We compare the "worth" with different metrics
    
    \begin{itemize}
        \item M1 : The sum of the absolute value of the in an out weights on the hidden-to-hidden connection
        \item M2 : The sum of the in an out weights on the hidden-to-hidden connection
        \item M3 : The absolute value of the sum of the in an out weights on the hidden-to-hidden connection
        \item M4 : The clustering index
        \item M5 : The absolute value of the ouput weights ( of the hidden to output matrix) 
        \item M6 : The absolute value of the input weights ( of the hidden to output matrix) 
        \item M7 : The sum of the two previous metrics.
    \end{itemize}

\end{frame}


\begin{frame}
    \frametitle{Comparaison with classical metrics}
    Backpropagation = Gradient descent on loss. 

    \begin{itemize} 
        \item use of the derivative of composition product to go back layer after layer...
        \item backpropagation through time for RNNs
    \end{itemize}
    \begin{center}
         \begin{figure}
            \includegraphics[width=0.2\textwidth]{Figures/m1.png}
            \hspace{0.1cm} \includegraphics[width=0.2\textwidth]{Figures/m2.png}
            \hspace{0.1cm} \includegraphics[width=0.2\textwidth]{Figures/m3.png}
            \hspace{0.1cm} \includegraphics[width=0.2\textwidth]{Figures/m4.png}
        \end{figure}
        \begin{figure}
            \includegraphics[width=0.2\textwidth]{Figures/m5.png}
            \hspace{1cm} \includegraphics[width=0.2\textwidth]{Figures/m6.png}
            \hspace{1cm} \includegraphics[width=0.2\textwidth]{Figures/m7.png}
        \end{figure}
    \end{center}

\end{frame}


\begin{frame}
    \frametitle{Comparaison with classical metrics}

\begin{figure}[!htb]
    \centering
    \begin{tabular}{|l|l|}
        \hline
        M1 & 3118 \\ \hline
        M2 & 3294 \\ \hline
        M3 & 3512 \\ \hline
        M4 & 3388 \\ \hline
        M5 & 3244 \\ \hline
        M6 & \textbf{2944} \\ \hline
        M7 & 3402 \\
        \hline
    \end{tabular}
    \label{fig:metrics}
    \rule{35em}{0.5pt}
    \caption[Table presenting the value of d for the different metrics ]{Table presenting the value of d for the different metrics}
\end{figure}

Results: None of the natural metrics have satisfying results.

\end{frame}


\begin{frame}
    \frametitle{Training : Backpropagation}
    Backpropagation = Gradient descent on loss. 

    \begin{itemize} 
        \item use of the derivative of composition product to go back layer after layer...
        \item backpropagation through time for RNNs
    \end{itemize}
    \begin{center}
        \begin{figure}
            \includegraphics[width=0.4\textwidth]{Figures/network.jpeg} 
        \end{figure}
    \end{center}

\end{frame}


\begin{frame}
    \frametitle{Training : Backpropagation}
    Backpropagation = Gradient descent on loss. 

    \begin{itemize} 
        \item use of the derivative of composition product to go back layer after layer...
        \item backpropagation through time for RNNs
    \end{itemize}
    \begin{center}
        \begin{figure}
            \includegraphics[width=0.4\textwidth]{Figures/network.jpeg} 
        \end{figure}
    \end{center}

\end{frame}

   

\section{Self Growing Recurrent Neural Networks}

\section{Application on natural problem : music generation}

\begin{frame}
    \begin{center}
    Thank you
    \end{center}
\end{frame}

%\frame[shrink=30] 
%{ 
%    \frametitle{References} 
%    \bibliographystyle{alpha} 
%    \bibliography{Bibliography.bib} 
%}

\end{document}
